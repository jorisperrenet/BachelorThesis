\chapter{Matrices from the example calculation}
\label{app:B}


%%% THIS IS WITHOUT SHIFTING
% \begin{align}
%     b_{ij} &= \begin{bmatrix}
%         0.68896 & 0.4346 & -0.02944 & -0.0097228 \\
%         0.68896 & 0.4346 & -0.02944 & -0.0097228 \\
%         0.42002 & 0.16566 & -0.11909 & -0.019683 \\
%         0.43745 & 0.20053 & -0.095841 & -0.014518 \\
%         0.43745 & 0.20053 & -0.095841 & -0.0025022 \\
%         0.39041 & 0.29461 & -0.15856 & 0.011436 \\
%         0.21126 & 0.47376 & -0.21828 & 0.018071 \\
%         0.21126 & 0.47376 & -0.21828 & 0.018071
%     \end{bmatrix} \,, \nonumber \\
%     c_{ik} &= \begin{bmatrix}
%         0.90797 & -0.12475 & -0.074376 & 0.018042 & 0.0044396 & 0.00021437 \\
%         1.0044 & -0.026695 & -0.034541 & 0.026119 & 0.0052566 & 0.00024732 \\
%         0.98878 & -0.052314 & -0.051575 & 0.020379 & 0.0042759 & 0.0001794 \\
%         0.98777 & -0.055127 & -0.054793 & 0.018459 & 0.003669 & 9.6965e-05 \\
%         0.98777 & -0.055268 & -0.053976 & 0.016086 & 0.0059934 & -0.001363 \\
%         0.89761 & 0.18189 & -0.30059 & 0.14164 & -0.024691 & 0.0013866 \\
%         1.547 & -0.76555 & 0.24232 & -0.009532 & -0.0046052 & 0.00040516 \\
%         2.204 & -1.5504 & 0.61913 & -0.10037 & 0.0063825 & -0.00012811
%     \end{bmatrix}\,.\nonumber
% \end{align}
% Therefore we can calculate $d_{ij}$

% \begin{equation}
%     d_{ij} &= \begin{bmatrix}
%         -0.85182 & 4.581 & -24.779 & 134.77 \\
%         0.14203 & -0.40212 & 1.0446 & -2.2333 \\
%         1.0466 & -2.2663 & 5.0931 & -11.85 \\
%         1.4641 & -1.0826 & 1.0714 & -1.1961 \\
%         1.3856 & 1.0132 & 1.0001 & 1.1164 \\
%         1.2855 & 2.9006 & 6.7888 & 16.416 \\
%         1.4454 & 5.4406 & 20.747 & 80.113 \\
%         1.3892 & 7.2533 & 38.127 & 201.75
%     \end{bmatrix}\,.\nonumber
% \end{equation}


% \section{Matrices $b_{ij}$, $c_{ij}$ and $d_{ij}$}
\label{app:matrices}
These are the matrices from the example calculation in Section \ref{examples} for the separate interpolation method described in Section \ref{sep_interpol}.
The computer code for generating these matrices can be found in Appendix \ref{app:code}.

\vfill
\vfill
\vfill
\begin{equation}
    B = \begin{bmatrix}
        -0.87832 & -0.26219 & 0.14557 & -0.0097228 \\
        -0.97689 & 0.10889 & 0.10182 & -0.0097228 \\
        -0.61727 & 0.34872 & 0.058066 & -0.019683 \\
        -0.02998 & 0.39005 & -0.03051 & -0.014518 \\
        0.43745 & 0.20053 & -0.095841 & -0.0025022 \\
        0.51415 & -0.10389 & -0.1071 & 0.011436 \\
        0.15594 & -0.348 & -0.055639 & 0.018071 \\
        -0.43025 & -0.39293 & 0.025681 & 0.018071
    \end{bmatrix} \,, \nonumber
\end{equation}
\vfill
\begin{equation}
    C = \begin{bmatrix}
        -0.83134 & 0.26959 & 0.096787 & -0.011336 & -0.0019914 & 0.00021437 \\
        -0.2559 & 0.46198 & 0.026152 & -0.018416 & -0.00030821 & 0.00024732 \\
        0.43407 & 0.41822 & -0.052521 & -0.014786 & 0.0015849 & 0.0001794 \\
        0.90271 & 0.18678 & -0.091602 & -0.0013727 & 0.0029417 & 0.000096965 \\
        0.98777 & -0.055268 & -0.053976 & 0.016086 & 0.0059934 & -0.001363 \\
        0.85769 & -0.062041 & 0.05026 & 0.024694 & -0.014291 & 0.0013866 \\
        0.89925 & 0.097732 & 0.017243 & -0.028331 & 0.0014721 & 0.00040516 \\
        0.99955 & -0.011708 & -0.077081 & -0.011424 & 0.0035001 & -0.00012811
    \end{bmatrix}\,,\nonumber
\end{equation}
\vfill
\begin{equation}
    D = \begin{bmatrix}
        -0.8518 & -0.52995 & -0.47276 & -0.49211 \\
        0.14199 & 0.23699 & 0.3016 & 0.38253 \\
        1.0466 & 0.87334 & 0.91425 & 1.0546 \\
        1.4641 & 1.1135 & 1.1177 & 1.2588 \\
        1.3856 & 1.0132 & 1.0001 & 1.1164 \\
        1.2855 & 0.97247 & 0.97915 & 1.1071 \\
        1.4454 & 1.1042 & 1.1126 & 1.2566 \\
        1.3901 & 1.0027 & 0.97905 & 1.0837
    \end{bmatrix}\,.\nonumber
\end{equation}
\vfill
\vfill
\vfill
